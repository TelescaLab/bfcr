\documentclass[useAMS,referee,usenatbib]{biom}
\usepackage{xcolor}
\usepackage{amsmath}
\usepackage{amsfonts}
\usepackage{amsmath}
\usepackage{graphicx}
\usepackage{multirow}
\usepackage{bbm}
\DeclareMathOperator{\E}{\mathbb{E}}
\newcommand{\vect}{\text{vec}}

\title[This is an Example of Recto Running Head]{Bayesian Functional Covariance Regression}

%  Here are examples of different configurations of author/affiliation
%  displays.  According to the Biometrics style, in some instances,
%  the convention is to have superscript *, **, etc footnotes to indicate 
%  which of multiple email addresses belong to which author.  In this case,
%  use the \email{ } command to produce the emails in the display.

%  In other cases, such as a single author or two authors from 
%  different institutions, there should be no footnoting.  Here, use
%  the \emailx{ } command instead. 

%  The examples below corrspond to almost every possible configuration
%  of authors and may be used as a guide.  For other configurations, consult
%  a recent issue of the the journal.

%  Single author -- USE \emailx{ } here so that no asterisk footnoting
%  for the email address will be produced.

%\author{John Author\emailx{email@address.edu} \\
%Department of Statistics, University of Warwick, Coventry CV4 7AL, U.K.}

%  Two authors from the same institution, with both emails -- use
%  \email{ } here to produce the asterisk footnoting for each email address

%\author{John Author$^{*}$\email{author@address.edu} and
%Kathy Authoress$^{**}$\email{email2@address.edu} \\
%Department of Statistics, University of Warwick, Coventry CV4 7AL, U.K.}

%  Exactly two authors from different institutions, with both emails  
%  USE \emailx{ } here so that no asterisk footnoting for the email address
%  is produced.

\author{John Shamshoian$^{1}$\email{donatello.telesca@ucla.edu}, Damla {\c S}ent{\"u}rk$^{1}$, Shafali Jeste$^{2}$, and Donatello Telesca$^{1,*}$ \\$^{1}$Department of Biostatistics, University of California, Los Angeles, California 90095, U.S.A.\\$^{2}$ Department of Psychiatry and Biobehavioral Sciences, University of California, Los Angeles,\\ California 90095, U.S.A.}
\title[Supporting Information for ``Bayesian Functional Covariance Regression'']{Supporting Information for ``Bayesian Functional Covariance Regression''}


\begin{document}


\label{firstpage}
\maketitle


\section{Web Appendix A: Markov-chain monte carlo sampling algorithm}
\label{s:mcmc}
In this section we give a detailed Markov-Chain Monte Carlo (MCMC) algorithm to sample from the posterior. Let $N$ be the number of independent functional responses and assume all response functions are observed on a common grid $T = \{t_{1}, \ldots, t_{n}\}$. Let $B$ be an $n\times p$ matrix with $B_{ij} = b_{j}(t_{i})$. Let $X$ be an $N\times r(d_{1})$ matrix with row $i$ equal to $\bmath{b}^{x}(\bmath{x}_{i})$. Let $Y$ be an $N\times n$ matrix with $Y_{ij} = y_{i}(t_{j})$, so that each row represents one discretized functional response. Let $\Gamma_{j}$ be an $N\times N$ diagonal matrix with $r$th diagonal element equal to $\eta_{rj}$ for $j = 1,\ldots,k$. 
\begin{enumerate}
	\item Update $\beta$:\\
	Let $\Omega_{r} = \tau_{1xr}\tilde{K}_{r} + \tau_{1tr}\tilde{K}$ if $p_{r} > 1$. Otherwise set $\Omega_{r} = \tau_{1tr}K$. Construct\\ $\Omega = \text{blkdiag}(\Omega_{1}, \ldots, \Omega_{R})$.\\
	Let $C = \sigma^{-2}X^{\top}X\otimes B^{\top}B + \Omega$\\
	Let $A = \sigma^{-2}\vect[\{B^{\top}Y^{\top} - B^{\top}B(\sum_{j=1}^{k}\Lambda_{j}X^{\top}\Gamma_{j})\}X]$\\
	Sample $\vect(\beta) \sim N(C^{-1}A, C^{-1})$
	
	\item Update $\Lambda_{j}$:\\
	Let $\Omega_{r} = \tau_{2xr}\tilde{K}_{r} + \tau_{2tr}\tilde{K} + \tau^{*}_{rj}\phi_{rj}$ if $p_{r} > 1$. Otherwise set $\Omega_{r} = \tau_{2tr}K +\tau^{*}_{rj}\phi_{rj}$.\\
	Construct $\Omega = \text{blkdiag}(\Omega_{1}, \ldots, \Omega_{R})$\\
	Let $C = \sigma^{-2}X^{\top}\Gamma_{j}^{2}\,X \otimes B^{\top}B + \Omega$\\
	Let $A = \sigma^{-2}\vect[\{B^{\top}Y^{\top}-B^{\top}B(\beta + \sum_{j'\neq j}\Lambda_{j'}X^{\top}\Gamma_{j'})\}\Gamma_{j}X]$\\
	Sample $\Lambda_{j} \sim N(C^{-1}A, C^{-1})$
	
	\item Update $\eta_{ij}$:\\
	Let $\bmath{\eta}_{i} = (\eta_{i1}, \ldots, \eta_{ik})$.\\
	Let $\ddot{X}_{i}$ be a $p\times k$ matrix with column $j$ equal to $\Lambda_{j}\bmath{b}^{x}(\bmath{x}_{i})$.\\
	Let $C = \sigma^{-2}\ddot{X}_{i}^{\top}B^{\top}B\ddot{X}_{i} + I_{k}$, where $I_{k}$ is the $k\times k$ identity matrix.\\
	Let $A = \sigma^{-2}\ddot{X}_{i}^{\top} \{B^{\top}Y_{i\cdot} - B^{\top}B\beta \bmath{b}^{x}(\bmath{x}_{i})\}$\\
	Sample $\bmath{\eta}_{i} \sim N(C^{-1}A, C^{-1})$
	
	\item Update $\sigma^{2}$:\\
	Let $A =Y^{\top} - B\beta X^{\top} + \sum_{j=1}^{k}B\Lambda_{j}X^{\top}\Gamma_{j}$\\
	Sample $\sigma^{-2} \sim \text{Gamma}(Nn/2+a_{\epsilon}, A \odot A/2 +b_{\epsilon})$, where $\odot$ denotes element-wise multiplication.
	
	\item Update $\tau_{1tr}$, $\tau_{1xr}$, $\tau_{2tr}$, $\tau_{2xr}$:\\
	Sample $\tau_{1tr} \sim \text{Gamma}\{\text{rank}(\tilde{K})/2 - 0.5, \vect(\beta_{r})^{\top}\tilde{K}\vect(\beta_{r})/2\}$\\
	Sample $\tau_{1xr} \sim \text{Gamma}\{\text{rank}(\tilde{K}_{r})/2 - 0.5, \vect(\beta_{r}^{\top})\tilde{K}_{r}\vect(\beta_{r})/2\}$ (if $p_{r} > 1$)\\
	Sample $\tau_{2trj} \sim \text{Gamma}\{\text{rank}(\tilde{K})/2 - 0.5, \vect(\Lambda_{rj})^{\top}\tilde{K}\vect(\Lambda_{rj})/2\}$\\
	Sample $\tau_{2xrj} \sim \text{Gamma}\{\text{rank}(\tilde{K}_{r})/2 - 0.5, \vect(\Lambda_{rj}^{\top})\tilde{K}_{r}\vect(\Lambda_{rj})/2\}$ (if $p_{r} > 1$)
	
	\item Update $\phi_{rj}$:\\
	Let $\phi_{irj}$ denote the $i$th diagonal element of $\phi_{rj}$. \\
	Let $\lambda_{irj}$ be the $i$th element of $\vect(\Lambda_{rj})$.\\
	Sample $\phi_{irj} \sim \text{Gamma}(a_{\phi} + 0.5, \tau^{*}_{rj}\lambda_{irj}^{2}/2 + b_{\phi})$
	
	\item Update $\delta_{r1}$:\\
	Let $A = \vect(\Lambda_{r1}^{\top})\phi_{r1}\vect(\Lambda_{r1})$\\
	Let $B = \sum_{j=2}^{k}\tau^{*}_{rj} \vect(\Lambda_{rj})^{\top}\phi_{rj}\vect(\Lambda_{rj})$\\
	Sample $\delta_{r1} \sim \text{Gamma}\{k p_{r} p/2 + a_{r0}, (A + B)/2+ 1\}$
	
	\item Update $\delta_{rj}$: \\
	Let $A = \sum_{j'=1}^{k}\tau^{*(j)}_{rj'}\vect(\Lambda_{rj'})^{\top}\phi_{rj'}\vect(\Lambda_{rj})$, where $\tau^{*(j)}_{rj'} = \tau^{*}_{rj'}$ if $j\neq j'$ and 1 otherwise.\\
	Sample $\delta_{rj'} \sim \text{Gamma}\{p_{r}p(k-j'+1)/2 + a_{r1}, A/2 + 1\}$
	
	\item Update $a_{r0}$:\\
	Let $\text{Gamma}^{*}(x,a,b)$ denote the Gamma density evaluated at $x$ with shape $a$ and rate $b$.\\
	Let $\phi(x)$ denote the standard normal cumulative distribution function evaluated at $x$.
	Sample candidate $a_{r0}^{*}\sim N(a_{r0}, 1)$ until $a_{r0}^{*} > 0$.\\
	Compute $A = \displaystyle\frac{\text{Gamma}^{*}(\delta_{r1}, a_{r0}^{*}, 1)\cdot\text{Gamma}^{*}(a^{*}_{r0}, 2, 1)\cdot \phi(a_{r0})}{\text{Gamma}^{*}(\delta_{r1}, a_{r0}, 1)\cdot\text{Gamma}^{*}(a_{r0}, 2, 1)\cdot \phi(a_{r0}^{*})}$\\
	Sample $U \sim \text{Unif}(0, 1)$\\
	If $U\leq A$, accept candidate $a_{r0}^{*}$. 
	
	\item Update $a_{r1}$:\\
	Let $\delta_{r2}^{*} = \prod_{j=2}^{k}\delta_{rj}$\\
	Let $\text{Gamma}^{*}(x,a,b)$ denote the Gamma density evaluated at $x$ with shape $a$ and rate $b$.\\
	Let $\phi(x)$ denote the standard normal cumulative distribution function evaluated at $x$.
	Sample candidate $a_{r1}^{*}\sim N(a_{r1}, 1)$ until $a_{r1}^{*} > 0$.\\
	Compute $A = \displaystyle\frac{\text{Gamma}^{*}(\delta_{r2}^{*}, a_{r1}^{*}, 1)\cdot\text{Gamma}^{*}(a^{*}_{r1}, 2, 1)\cdot \phi(a_{r1})}{\text{Gamma}^{*}(\delta_{r2}^{*}, a_{r1}, 1)\cdot\text{Gamma}^{*}(a_{r1}, 2, 1)\cdot \phi(a_{r1}^{*})}$\\
	Sample $U \sim \text{Unif}(0, 1)$\\
	If $U\leq A$, accept candidate $a_{r1}^{*}$. 
	
	\item Update missing values of $Y$:\\
	Suppose $y_{i}(t_{j})$ is missing.\\
	Let $\mu = \bmath{b}(t_{j})\beta\bmath{b}^{x}(\bmath{x}_{i}) + \sum_{j=1}^{k}\bmath{b}(t_{j})\Lambda_{j}\bmath{b}^{x}(\bmath{x}_{i})\eta_{ij}$\\
	Sample $y_{i}(t_{j}) \sim N(\mu, \sigma^{2})$
\end{enumerate}
For convenience, Table contains notation used in the main article and this document of supporting information.

\begin{table}
	\caption{Notation used in the main article and document of supporting information.}
	\label{t:notation}
	\begin{center}
		\begin{tabular}{lll}
			Name & Description & Dimension\\
			\hline
			$\bmath{x}_{i}$ & Covariate vector for the $i$th functional response & $d_{1}\times 1$ \\
			$\bmath{x}_{ir}$ & $r$th group of covariates associated with the $i$th functional response&\\
			$\bmath{b}^{r}(\bmath{x}_{ir})$ & $r$th group of covariates expanded into basis functions evaluated at $\bmath{x}_{ir}$&$p_{r}\times 1$\\
			$\bmath{b}^{x}(\bmath{x}_{i})$ & Concatenated $\bmath{b}^{r}(\bmath{x}_{ir})$, $r=1,\ldots,R$ & $r(d_{1})\times 1$\\
			$p_{r}$ & \# of basis functions for $r$th covariate group expansion&\\
			$r(d_{1})$ & Total number of covariate basis functions used equal to $\sum_{r=1}^{R}p_{r}$&\\
			$\bmath{b}(t)$ & Basis functions for functional dimension evaluated at $t$ & $p\times 1$\\
			$\mu(t, \bmath{x}_{i})$ & Covariate-adjusted functional mean equal to $\bmath{b}(t)^{\top}\beta\bmath{b}^{x}(\bmath{x}_{i})$ &\\
			$\beta$ & Fixed effect parameter matrix equal to &
			%$\beta_{r}$ & Fixed effect &\\
			%$R$ & Number of 
		\end{tabular}
	\end{center}
\end{table}
\section{Web Appendix B: Additional details on posterior inference}
\section{Web Appendix C: Setting hyperparameters}
\section{Web Appendix D: Extended simulation results}
\section{Web Appendix E: Additional details on case studies}
%\cite{Aguilera2013}
%\bibliographystyle{biom}  
%\bibliography{mybiblio}


\end{document}
