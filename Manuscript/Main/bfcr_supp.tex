\documentclass[useAMS,referee,usenatbib]{biom}
\usepackage{xcolor}
\usepackage{amsmath}
\usepackage{amsfonts}
\usepackage{amsmath}
\usepackage{graphicx}
\usepackage{multirow}
\usepackage{bbm}
\DeclareMathOperator{\E}{\mathbb{E}}
\newcommand{\vect}{\text{vec}}

\title[This is an Example of Recto Running Head]{Bayesian Functional Covariance Regression}

%  Here are examples of different configurations of author/affiliation
%  displays.  According to the Biometrics style, in some instances,
%  the convention is to have superscript *, **, etc footnotes to indicate 
%  which of multiple email addresses belong to which author.  In this case,
%  use the \email{ } command to produce the emails in the display.

%  In other cases, such as a single author or two authors from 
%  different institutions, there should be no footnoting.  Here, use
%  the \emailx{ } command instead. 

%  The examples below corrspond to almost every possible configuration
%  of authors and may be used as a guide.  For other configurations, consult
%  a recent issue of the the journal.

%  Single author -- USE \emailx{ } here so that no asterisk footnoting
%  for the email address will be produced.

%\author{John Author\emailx{email@address.edu} \\
%Department of Statistics, University of Warwick, Coventry CV4 7AL, U.K.}

%  Two authors from the same institution, with both emails -- use
%  \email{ } here to produce the asterisk footnoting for each email address

%\author{John Author$^{*}$\email{author@address.edu} and
%Kathy Authoress$^{**}$\email{email2@address.edu} \\
%Department of Statistics, University of Warwick, Coventry CV4 7AL, U.K.}

%  Exactly two authors from different institutions, with both emails  
%  USE \emailx{ } here so that no asterisk footnoting for the email address
%  is produced.

\author{John Shamshoian$^{1}$\email{donatello.telesca@ucla.edu}, Damla {\c S}ent{\"u}rk$^{1}$, Shafali Jeste$^{2}$, and Donatello Telesca$^{1,*}$ \\$^{1}$Department of Biostatistics, University of California, Los Angeles, California 90095, U.S.A.\\$^{2}$ Department of Psychiatry and Biobehavioral Sciences, University of California, Los Angeles,\\ California 90095, U.S.A.}
\title[Supporting Information for ``Bayesian Functional Covariance Regression'']{Supporting Information for ``Bayesian Functional Covariance Regression''}


\begin{document}


\label{firstpage}
\maketitle


\section{Web Appendix A: Markov-Chain Monte Carlo Sampling Algorithm}
\label{s:mcmc}
In this section we give a detailed Markov-Chain Monte Carlo (MCMC) algorithm to sample from the posterior. Let $N$ be the number of independent functional responses and assume all response functions are observed on a common grid $T = \{t_{1}, \ldots, t_{n}\}$. Let $B$ be an $n\times p$ matrix with $B_{ij} = b_{j}(t_{i})$. Let $X$ be an $N\times r(d_{1})$ matrix with row $i$ equal to $\tilde{\bmath{X}}(\bmath{x}_{i})$. Let $Y$ be an $N\times n$ matrix with $Y_{ij} = y_{i}(t_{j})$, so that each row represents one discretized functional response. Let $\Gamma_{j}$ be an $N\times N$ diagonal matrix with $r$th diagonal element equal to $\eta_{rj}$ for $j = 1,\ldots,k$. 
\begin{enumerate}
	\item Update $\beta$:\\
	Let $\Omega_{r} = \tau_{1xr}\tilde{K}_{r} + \tau_{1tr}\tilde{K}$ if $p_{r} > 1$. Otherwise set $\Omega_{r} = \tau_{1tr}K$. Construct\\ $\Omega = \text{blkdiag}(\Omega_{1}, \ldots, \Omega_{R})$.\\
	Let $C = \sigma^{-2}X^{\top}X\otimes B^{\top}B + \Omega$\\
	Let $A = \sigma^{-2}\vect[\{B^{\top}Y^{\top} - B^{\top}B(\sum_{j=1}^{k}\Lambda_{j}X^{\top}\Gamma_{j})\}X]$\\
	Sample $\vect(\beta) \sim N(C^{-1}A, C^{-1})$
	
	\item Update $\Lambda_{j}$:\\
	Let $\Omega_{r} = \tau_{2xr}\tilde{K}_{r} + \tau_{2tr}\tilde{K} + \tau^{*}_{rj}\phi_{rj}$ if $p_{r} > 1$. Otherwise set $\Omega_{r} = \tau_{2tr}K +\tau^{*}_{rj}\phi_{rj}$.\\
	Construct $\Omega = \text{blkdiag}(\Omega_{1}, \ldots, \Omega_{R})$\\
	Let $C = \sigma^{-2}X^{\top}\Gamma_{j}^{2}\,X \otimes B^{\top}B + \Omega$\\
	Let $A = \sigma^{-2}\vect[\{B^{\top}Y^{\top}-B^{\top}B(\beta + \sum_{j'\neq j}\Lambda_{j'}X^{\top}\Gamma_{j'})\}\Gamma_{j}X]$\\
	Sample $\Lambda_{j} \sim N(C^{-1}A, C^{-1})$
	
	\item Update $\eta_{ij}$, $j=1,\ldots,k$:\\
	Let $\bmath{\eta}_{i} = (\eta_{i1}, \ldots, \eta_{ik})$.\\
	Let $\ddot{X}_{i}$ be a $p\times k$ matrix with column $j$ equal to $\Lambda_{j}\tilde{\bmath{X}}(\bmath{x}_{i})$.\\
	Let $C = \sigma^{-2}\ddot{X}_{i}^{\top}B^{\top}B\ddot{X}_{i} + I_{k}$, where $I_{k}$ is the $k\times k$ identity matrix.\\
	Let $A = \sigma^{-2}\ddot{X}_{i}^{\top} \{B^{\top}Y_{i\cdot} - B^{\top}B\beta \tilde{\bmath{X}}(\bmath{x}_{i})\}$\\
	Sample $\bmath{\eta}_{i} \sim N(C^{-1}A, C^{-1})$
	
	\item Update $\sigma^{2}$:\\
	
\end{enumerate}
%\cite{Aguilera2013}
%\bibliographystyle{biom}  
%\bibliography{mybiblio}


\end{document}
